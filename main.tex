\documentclass[10pt,a4paper,sans]{moderncv}

\moderncvstyle{classic}
\moderncvcolor{blue}

\usepackage[maxnames=10]{biblatex}
\usepackage[utf8]{inputenc}
\usepackage[scale=0.8]{geometry}

\input{personaldata.tex}
\bibliography{publications}

\setlength{\hintscolumnwidth}{3cm}

\makeatletter
\name{Gianluca}{Brian}
\title{Curriculum Vitae}
\email{\@@@email}
\phone[mobile]{\@@@phone}
\extrainfo{
    Date of birth: 05/06/1994
}
\social[github][github.com/mclytar]{github.com/mclytar}
\makeatother

\makeatletter\renewcommand*{\bibliographyitemlabel}{\@biblabel{\arabic{enumiv}}}\makeatother

\def\pplus{\texttt{++}}
\def\sharp{\texttt{\#}}

\renewcommand{\refname}{Research activities and publications}
\DefineBibliographyExtras{english}{\let\finalandcomma=\empty}
\newcommand{\pubitem}[1]{
    \cvitem{\citefield{#1}{year}}{\textbf{\citefield{#1}{title}} (with \citename{#1}{author}), in \textit{\citefield{#1}{booktitle} } }
}

\begin{document}
    \makecvtitle
    \section{Research interests}
    \cvitem{}{
        I am interested in theoretical and applied cryptography.
        Current topics include: leakage-resilience, non-malleability, secret sharing.
    }

    \section{Current position}
    \cventry{Nov 21 -- Nov 22}{Visiting Researcher}{University of Warsaw, Poland}{}{}{}
    \cvitem{\textbf{Promoters}}{Prof. Stefan Dziembowski}
    \cventry{Nov 19 -- Now}{Ph.D. student in Computer Science}{Sapienza University of Rome, Italy}{}{}{}
    \cvitem{\textbf{Advisor}}{Prof. Daniele Venturi}
    \section{Ph.D. Schools attended}
    \cventry{Apr 22}{IACR-Crossing School}{La Valletta, Malta}{}{}{
        Main topic: \emph{Combinatorial Techniques in Cryptography}
    }
    \cventry{Jul -- Aug 21}{MSRI Summer Graduate School}{Virtual on Zoom/Sococo}{}{}{
        Main topic: \emph{Foundations and Frontiers of Probabilistic Proofs}
    }
    \cventry{Feb 21}{BIU Winter School on Cryptography}{Virtual on Zoom}{}{}{
        Main topic: \emph{Cryptography in a Quantum World}
    }
    \cventry{Feb 20}{BIU Winter School on Cryptography}{Tel Aviv, Israel}{}{}{
        Main topic: \emph{Information Theoretic Cryptography}
    }

    \section{Education}
    \cventry{Sep 16 -- Jul 19}{M. Sc. Mathematics}{Sapienza University of Rome, Italy}{}{}{
        Final grade: \emph{106/110}\\
        Thesis title: \emph{Non-malleable Secret Sharing under Selective Partitioning}.\\
        Advisors: \emph{Prof. Claudia Malvenuto and Prof. Daniele Venturi}.\\
        Description: \emph{We revisit bounded leakage-resilient secret sharing under selective partitioning, adding non-malleability at the cost of a slight loss in bits for the leakage resilience bound}.
    }
    \cventry{Sep 13 -- Dec 16}{B. Sc. Mathematics}{Sapienza University of Rome, Italy}{}{}{
        Final grade: \emph{105/110}\\
        Thesis title: \emph{Complexity and public key cryptosystems}.\\
        Advisor: \emph{Prof. Alessandro D'Andrea}.\\
        Description: \emph{A summary on the relation between hard-to-solve problems in mathematics (e.g. discrete logarithm and factorization) and public key cryptography}.
    }

    \nocite{*}
    \section{Publications}
    \pubitem{EUROCRYPT:BFMV22}
    \pubitem{ToSC:BFRV22}
    \pubitem{TCC:BFV21}
    \pubitem{EUROCRYPT:BFORSSV21}
    \pubitem{CRYPTO:BFOSV20}
    \pubitem{TCC:BFV19}

    \section{Selected Talks}
    \cventry{2020}{Leakage-Resilient Non-Malleable Secret Sharing}{Seminar at De Cifris Schola Latina}{Virtual on Google Meet}{October 2020}{}

    \section{Computer skills}
    \cvitem{Programming Languages}{
        Fluent in Rust.
        Good knowledge of C, C\pplus{}, Intel x86 Assembly.
        Basic knowledge of Visual Basic, Python, Kotlin and C\sharp{}, in no particular order.
    }
    \cvitem{Typesetting}{
        \LaTeX, including Tikz and the \texttt{beamer} template for slides.
    }
    \cvitem{Software}{
        Microsoft Office suite (Word, Excel, Power Point), JetBrains CLion, MS Visual Studio, MS Visual Studio Code, Git, Powershell.
    }
    \cvitem{Misc.}{
        Arduino (both programming and circuit design).
    }
    \cvitem{Operating systems}{
        Windows, GNU/Linux
    }

    \section{Languages}
    \cvitem{Italian}{Mother tongue.}
    \cvitem{English}{Self-assessment (Common European Framework of Reference for Languages, CEFR).}
    \cvitem{}{\begin{tabular}{l@{\hspace{.5cm}}l}
        Reading: & C1 (Proficient user)\\
        Listening: & C1 (Proficient user)\\
        Writing: & C1 (Proficient user)\\
        Speaking: & C1 (Proficient user)
    \end{tabular}}

    \section{Spare time activities}
    \cvitem{Mathematics}{
        I participated in several mathematics challenges and olympic games (both alone and in team) during my high school years.
        %also winning the semifinal of ``Mathematics Games'' (``Giochi matematici'') and participating in the national final at Bocconi University of Milan.
    }
    \cvitem{Hobbies}{
        Videogames: mainly strategic and management videogames or racing simulations.
    }
    \cvitem{}{
        Misc.: I am building an Intel 8086 computer by studying the specifications of the various components
        and designing both hardware (motherboard, VGA) and software (BIOS, Kernel, OS, Utility software, C compiler, maybe a custom programming language, etc.).
    }
    \cvitem{Interests}{Computers, math and science in general.}

\end{document}
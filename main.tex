\documentclass[10pt,a4paper,sans]{moderncv}

\moderncvstyle{classic}
\moderncvcolor{blue}

\usepackage[maxnames=10]{biblatex}
\usepackage[utf8]{inputenc}
\usepackage[scale=0.8]{geometry}

\input{personaldata.tex}
\bibliography{publications}

\makeatletter
\name{Gianluca}{Brian}
\title{Curriculum Vitae}
\email{\@@@email}
\phone[mobile]{\@@@phone}
\extrainfo{Date of birth: 05/06/1994}
\social[github][github.com/mclytar]{github.com/mclytar}
\makeatother

\makeatletter\renewcommand*{\bibliographyitemlabel}{\@biblabel{\arabic{enumiv}}}\makeatother

\def\pplus{\texttt{++}}
\def\sharp{\texttt{\#}}

\renewcommand{\refname}{Research activities and publications}
\DefineBibliographyExtras{english}{\let\finalandcomma=\empty}
\newcommand{\pubitem}[1]{
    \cvitem{\citefield{#1}{year}}{\textbf{\citefield{#1}{title}} (with \citename{#1}{author}), in \textit{\citefield{#1}{booktitle} } }
}

\begin{document}
    \makecvtitle

    \section{Current position}
    \cventry{2019--Now}{PhD student in Computer Science}{Sapienza University of Rome, Italy}{}{}{
        Research topic: Cryptography\\
        Advisor: \emph{Prof. Daniele Venturi}.
    }
    \subsection{PhD Schools attended}
    \cventry{07/2021}{MSRI Summer Graduate School}{Virtual on Zoom/Sococo}{}{}{
        Main topic: \emph{Foundations and Frontiers of Probabilistic Proofs}
    }
    \cventry{02/2021}{BIU Winter School on Cryptography}{Virtual on Zoom}{}{}{
        Main topic: \emph{Cryptography in a Quantum World}
    }
    \cventry{02/2020}{BIU Winter School on Cryptography}{Tel Aviv, Israel}{}{}{
        Main topic: \emph{Information Theoretic Cryptography}
    }

    \section{Education}
    \cventry{2016--2019}{Master Degree in Mathematics}{Sapienza University of Rome, Italy}{}{}{
        Final mark: \emph{106/110}\\
        Thesis title: \emph{Non-malleable Secret Sharing under Selective Partitioning}.\\
        Advisors: \emph{Prof. Claudia Malvenuto and Prof. Daniele Venturi}.\\
        Description: \emph{We revisit bounded leakage-resilient secret sharing under selective partitioning, adding non-malleability at the cost of a slight loss in bits for the leakage resilience bound}.
    }
    \cventry{2013--2016}{Bachelor Degree in Mathematics}{Sapienza University of Rome, Italy}{}{}{
        Final mark: \emph{105/110}\\
        Thesis title: \emph{Complexity and public key cryptosystems}.\\
        Advisor: \emph{Prof. Alessandro D'Andrea}.\\
        Description: \emph{A brief summary on the relation between hard-to-solve problems in mathematics (e.g. discrete logarithm and factorization) and public key cryptography}.
    }

    \nocite{*}
    \section{Publications}
    \pubitem{EUROCRYPT:BFORSSV21}
    \pubitem{CRYPTO:BFOSV20}
    \pubitem{TCC:BFV19}

    \section{Talks}
    \cventry{2020}{Leakage-Resilient Non-Malleable Secret Sharing}{Seminar at De Cifris Schola Latina}{Virtual on Google Meet}{October 2020}{}
    \cventry{2020}{Non-Malleable Secret Sharing against Bounded Joint-Tampering Attacks in the Plain Model}{Accepted paper at CRYPTO 2020}{Virtual on Zoom}{August 2020}{}
    \cventry{2019}{Continuously Non-malleable Secret Sharing for General Access Structures}{Accepted paper at TCC 2019}{Nuremberg, Germany}{December 2019}{}

    %\section{Personal projects and professional experience}
    %\cvitem{$\bullet$}{
    %    Development of a web application to help the organization of the annual Christmas Show in the Mathematics department of my university,
    %    with RESTful API and backend server in Rust, Actix-Web and Diesel-MySql.
    %}
    %\cvitem{$\bullet$}{
    %    Minor full-stack web development experience, mainly PHP\texttt{+}Apache\texttt{+}MySql backend and HTML\texttt{+}CSS\texttt{+}JS/jQuery frontend.
    %}

    %\section{Honors and awards}

    \section{Computer skills}
    \cvitem{Programming}{
        Fluent in Rust.
        Good knowledge of C, C\pplus{} and Intel x86 Assembly programming languages, JavaScript and PHP scripting languages,
        and, in no particular order, HTML, CSS/SASS, SQL, \LaTeX.
        Basic knowledge of Fortran, Microsoft Visual Basic, Python and C\sharp{} (along with .Net framework).
    }
    \cvitem{}{
        I also have experience with Arduino, both programming and circuit design.
    }
    \cvitem{Development tools}{
        Good knowledge of Git and the Git workflow.
        My main development tools include JetBrains IDEs (CLion, PhpStorm, WebStorm), Microsoft Visual Studio and Visual Studio Code, Powershell, Notepad\pplus.
    }
    \cvitem{Operating systems}{
        Good knowledge of the Windows operating system and, in particular, Windows XP/7/8/10.
        Basic knowledge of GNU/Linux and other UNIX-based OSes.
    }

    \section{Languages}
    \cvitem{Italian}{Mother tongue.}
    \cvitem{English}{Self-assessment (Common European Framework of Reference for Languages, CEFR).}
    \cvitem{}{\begin{tabular}{l@{\hspace{.5cm}}l}
        Reading: & C1 (Proficient user)\\
        Listening: & C1 (Proficient user)\\
        Writing: & C1 (Proficient user)\\
        Speaking: & C1 (Proficient user)
    \end{tabular}}

    \section{Spare time activities}
    \cvitem{Mathematics}{
        I participated in several mathematics challenges and olympic games (both alone and in team) during my high school years,
        also winning the semifinal of ``Mathematics Games'' (``Giochi matematici'') and participating in the national final at Bocconi University of Milan.
    }
    \cvitem{Hobbies}{
        Music: sometimes I play the electric guitar;
        I also like listening to Hi-Fi music.
    }
    \cvitem{}{Videogames: mainly strategic and management videogames or racing simulations.}
    \cvitem{}{Misc.: I am building an Intel 8086 computer, designing both the motherboard and the VGA card.}
    \cvitem{Interests}{Programming, math and science in general.}

\end{document}